\documentclass[10pt, technote]{article}
\usepackage{eadca-template}
\usepackage[plain]{algorithm}

\usepackage[brazil]{babel}
\usepackage[utf8]{inputenc}
\usepackage[T1]{fontenc}

\usepackage{graphicx,url}
\usepackage[hang]{subfigure}
\usepackage{psfrag}

\usepackage{siunitx}
\usepackage{mathtools}
\usepackage{booktabs}
\graphicspath{{../figures/}}

\sloppy

\title{{\noindent \includegraphics[scale = 0.5]{banner-grande.png}}\\ Predição de Séries Temporais Baseada em Redes Neurais
Artificiais}

\author{Aluno: João Pedro de Oliveira Pagnan\\Orientador: Prof.  Levy Boccato}

\address{Faculdade de Engenharia Elétrica e de Computação (FEEC) \\
  Universidade Estadual de Campinas (UNICAMP)}

\hyphenation{}
\pagestyle{fancy}

\begin{document}

\twocolumn[
\maketitle
\thispagestyle{fancy}
  \keywords{Redes Neurais Artificiais, Sistemas Caóticos, Modelos Preditores Não-Lineares, Séries Temporais}
\hrule
]

\section{Introdução}

\section{Metodologia}

\section{Resultados}

\section{Conclusões}

%\bibliographystyle{ieeetr}
%\bibliography{bib}

%\pdfinfo{
%   /Title  (Congresso PIBIC 2021 - João P. Pagnan)
%   /CreationDate (D:20040502195600)
%}


\end{document}